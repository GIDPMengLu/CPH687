\documentclass{article}
\usepackage[pdftex]{graphicx}
\usepackage{rotating}
\usepackage{amsmath}
\usepackage{authblk}
\usepackage{float}
\usepackage{bm}
\usepackage{multirow}
\usepackage{fullpage}
\usepackage{Sweave}
\usepackage{pdflscape}
\usepackage{amsmath}   
\usepackage{amssymb}
\usepackage{wasysym}
\usepackage{ctable}
\newcommand{\fg}[3]{\begin{figure}[htbp]%
        \leavevmode%
        \centerline{\includegraphics{#1}}%
        \caption{#3}%
        \label{#2}%
        \end{figure}}
\newcommand{\fgh}[4]{\begin{figure}[htbp]%
        \leavevmode%
        \centerline{\includegraphics{#1}}%
        \caption[#4]{#3}%
        \label{#2}%
        \end{figure}}
\begin{document} 
\Sconcordance{concordance:CPH687Homework2.tex:CPH687Homework2.Rnw:%
1 128 1}



\title{CPH 687 Homework 2}
%\author[1]{\small{Bruce Barber}}
\author{\small{Meng Lu}}

\affil{\footnotesize{GIDP Statistics \\ Email: menglu@email.arizona.edu}}
%\affil[2]{\footnotesize{Department of Pediatric, The University of Arizona}}
%\affil[2]{\footnotesize{Statistical Consulting Laboratory \\ email: zlu@arizona.edu}}

\maketitle
\section{Problem 1}
Question: \\
Suppose that $\bm{C}$ is nonsingular, show that the eigenvalues of
$\bm{C^{-1}AC}$ are the same as the eigenvalues of $\bm{A}$\\
Proof:\\
     \begin{align*}
       &\text{Assume that the eigenvalues of }\bm{C^{-1}AC}\text{ are }\lambda\\
       &\text{Then }\bm{C^{-1}ACx}=\lambda \bm{x}\\
       \therefore &\bm{ACx}=\lambda \bm{Cx}\\
       \therefore &\text{The eigenvalues of }\bm{A} \text{ are the same as
       the eigenvalues of }\bm{C^{-1}AC}
     .
     \end{align*}
     
    
            
            
\section{Problem 2}
Question:\\ 
If the eigenvalues of $\bm{A}$ satisfy $|\lambda_i|<1$ for all
$\bm{A}$ is diagonable, show that $(\bm{I-A})^{-1}=\Sigma_{i=0}^{\infty}A^i$\\
Proof:\\
   \begin{align*}
     \because &\bm{A} is diagonable\\
     \therefore \bm{A}&=\bm{T\Lambda T^{\prime}}\\
     \bm{A^i}&=\bm{T\Lambda^i T^{\prime}}\\
     \therefore \Sigma_{i=0}^{\infty}\bm{A^i}&=\bm{T}\Sigma_{i=0}^{\infty}\bm{\lambda^i}\bm{T^{i}}\\
     (\bm{I}-\bm{A})^{-1}&=(\bm{T}(\bm{I}-\bm{\Lambda})\bm{T^{\prime}})^{-1}\\
     &=\bm{T}(\bm{I}-\bm{\Lambda})^{-1}\bm{T^{\prime}}\\
     (\bm{I}-\bm{\Lambda})^{-1}&=(diag(1-\lambda_{1},1-\lambda_{2}, \cdots, 1-\lambda_{n}))^{-1}\\
     &=diag(\frac{1}{1-\lambda_{1}},\frac{1}{1-\lambda_{2}}, \cdots, \frac{1}{1-\lambda_{n}})\\
     \Sigma_{i=0}^{\infty}\bm{\Lambda}^{i}&=diag(\Sigma_{i=0}^{\infty}\lambda_{1}^{i},\Sigma_{i=0}^{\infty}\lambda_{2}^{i},
     \cdots, \Sigma_{i=0}^{\infty}\lambda_{n}^{i})\\
     &=diag(\frac{1}{1-\lambda_{1}},\frac{1}{1-\lambda_{2}}, \cdots, \frac{1}{1-\lambda_{n}})\\
     \therefore (\bm{I-\Lambda})^{-1}&=\Sigma_{i=0}^{\infty}\Lambda^i\\
     \therefore (\bm{I-A})^{-1}&=\Sigma_{i=0}^{\infty}A^i
     .
     \end{align*}
     
          
\section{Problem 3}
Question:\\
Let $Q(\bm{t})=\frac{t^{\prime}\bm{A}t}{t^{\prime}\bm{M}t}$, where
$\bm{M_{n\times n}}$ is positive definite and $\bm{A_{n\times n}}$ is
symmetric. Then, show that\\
\begin{align*}
  &\max_{\bm{t}\neq 0}Q(\bm{t})=\lambda_1 \text{ and }\min_{\bm{t}\neq 0}Q(\bm{t})=\lambda_n
  .
  \end{align*}\\
where $\lambda_1\geq \lambda_2 \geq \cdots \geq \lambda_n$ are the
eigenvalues of $\bm{M^{-1}A}$.\\
Proof:\\
\begin{align*}
  &\because \bm{M} \text{ is positive definite}\\
  &\therefore \text{There exists invertible matrix }\bm{C} \text{ such
    that }\bm{M}=\bm{C^{\prime}C}\\
  &\therefore Q(\bm{t})=\frac{t^{\prime}\bm{A}t}{t^{\prime}\bm{C^{\prime}C}t}\\
  &\text{Let }\bm{y}=\bm{Ct}\\
  &\text{Then }\bm{t}=\bm{C^{\prime}y}\\
  &\therefore Q(\bm{t})=Q(\bm{y})=\frac{\bm{y^{\prime}}(\bm{C^{-1}})^{\prime}\bm{A}\bm{C^{-1}y}}{\bm{y^{\prime}y}}\\
  &\therefore \max_{\bm{y}\neq 0}Q(\bm{y})=\alpha_1 \text{ and }\min_{\bm{y}\neq 0}Q(\bm{y})=\alpha_n\\
  &  \text{where }\alpha_1\geq \alpha_2 \geq \cdots \geq \alpha_n
  \text{ are the eigenvalues of }
(\bm{C^{-1}})^{\prime}\bm{A}\bm{C^{-1}}\\
&\because \text{the eigenvalues of }\bm{AB} \text{ are equal to the
  eigenvalues of }\bm{BA}\\
&\therefore \text{the eigenvalues of }(\bm{C^{-1}})^{\prime}\bm{A}\bm{C^{-1}} \text{ are equal to the
  eigenvalues of }\bm{C^{-1}} (\bm{C^{-1}})^{\prime}\bm{A}\\
  &\because \bm{C^{-1}} (\bm{C^{-1}})^{\prime}=\bm{M^{-1}}\\
  &\therefore \lambda_{i}=\alpha_i (i=1,2,\cdots,n)\\
  &\therefore \max_{\bm{t}\neq 0}Q(\bm{t})=\lambda_1 \text{ and }\min_{\bm{t}\neq 0}Q(\bm{t})=\lambda_n
  .
  \end{align*}

  
\section{Problem 4}
Question:\\
Given that $\bm{X_{n\times p}}$ has full rank, show $\bm{X^{\prime}X}$
is positive definite.\\
Proof:\\
\begin{align*}
  \because &\bm{X}\text{ has full rank}\\
  \therefore &\bm{Xt}\neq 0 (\bm{t}\neq 0)\\
  \therefore &\bm{t^{\prime}X^{\prime}Xt}=(\bm{Xt})^{\prime}(\bm{Xt})>0\\
  \therefore &\bm{X^{\prime}X}\text{ is positive definite}
  .
  \end{align*}

\end{document}
