\documentclass{article}
\usepackage[pdftex]{graphicx}
\usepackage{rotating}
\usepackage{amsmath}
\usepackage{authblk}
\usepackage{float}
\usepackage{bm}
\usepackage{multirow}
\usepackage{fullpage}
\usepackage{Sweave}
\usepackage{pdflscape}
\usepackage{amsmath}   
\usepackage{amssymb}
\usepackage{wasysym}
\usepackage{ctable}
\newcommand{\fg}[3]{\begin{figure}[htbp]%
        \leavevmode%
        \centerline{\includegraphics{#1}}%
        \caption{#3}%
        \label{#2}%
        \end{figure}}
\newcommand{\fgh}[4]{\begin{figure}[htbp]%
        \leavevmode%
        \centerline{\includegraphics{#1}}%
        \caption[#4]{#3}%
        \label{#2}%
        \end{figure}}
\begin{document} 
\Sconcordance{concordance:CPH687Homework1.tex:CPH687Homework1.Rnw:%
1 138 1}



\title{CPH 687 Homework 1}
\author{\small{Meng Lu}}
\affil{\footnotesize{GIDP Statistics \\ Email: menglu@email.arizona.edu}}
\maketitle

\section{Problem 1}
Question:\\
Proof the following. If $\bm{A}$ is an $n\times p$ matrix and $\bm{B}$
a $p\times q$ matrix, then the product $\bm{C}=\bm{AB}$ has the
following properties:\\
(a) Every column of $\bm{C}$ is a linear combination of columns of $\bm{A}$\\
(b) Every row of $\bm{C}$ is a linear combination of rows of $\bm{B}$\\
Proof:\\
(a)\\
\begin{align*}
  &\text{Let }\bm{e_j}\text{ be a }q\times 1 \text{ vector that the
  }jth\text{ position will be 1 the other places will be zero}\\
  \text{Then }\bm{c_j}&=\bm{C}\times \bm{e_j}\\
  &=\bm{AB}\times \bm{e_j}\\
  &=\bm{A}\times \bm{b_j}\\
  &=(a_1,a_2, \cdots, a_p)\bm{b_j}\\
  &=\Sigma_{i=1}^{p}a_ib_{ij}\\
  &\text{Every column of }\bm{C}\text{ is a linear combination of
    columns of }\bm{A}
  .
  \end{align*}\\
(b)\\
\begin{align*}
 &\text{Let }\bm{e_i}\text{ be a }1\times n \text{ vector that the }ith
 \text{ position will be 1 and the other places will be zero}\\
 \text{Then }\bm{c_i}&=\bm{e_i}\times \bm{C}\\
 &=\bm{e_i}\times \bm{AB}\\
 &=\bm{a_i}\times \bm{B}\\
 &=\bm{a_i}(b_1,b_2, \cdots, b_p)^{\prime}\\
 &=\Sigma_{i=1}^{p}a_{ij}b_j\\
 &\text{Every row of }\bm{C} \text{ is a linear combination of rows of
 }\bm{B}
 .
 \end{align*}
 
\section{Problem 2}
Question:\\
Show directly that if $\bm{A}$ is an $n \times p$ matrix and $\bm{B}$
is $p \times n$, then\\
\begin{align*}
  &(\bm{I_{n}-AB})^{-1}=\bm{I_{n}}+\bm{A}(\bm{I_{p}-BA})^{-1}\bm{B}
  .
  \end{align*}\\
provided the inverses exist. (This verifies a simplified version of
the Woodbury binomial inverse theorem.)\\
Proof:\\
\begin{align*}
  (\bm{I_n}-(\bm{AB}))(\bm{I_n}+\bm{A}(\bm{I_p}-\bm{BA})^{-1}\bm{B})&=\bm{I_n}+\bm{A}(\bm{I_p}-\bm{BA})^{-1}\bm{B}-\bm{AB}-\bm{ABA}(\bm{I_p}-\bm{BA})^{-1}B\\
  &=\bm{I_n}-\bm{AB}+\bm{A}[(\bm{I_p}-\bm{BA})^{-1}-\bm{BA}(\bm{I_p}-\bm{BA})^{-1}]\bm{B}\\
  &=\bm{I_n}-\bm{AB}+\bm{A}(\bm{I_p}-\bm{BA})(\bm{I_p}-\bm{BA})^{-1}\bm{B}\\
  &=\bm{I_n}-\bm{AB}+\bm{AB}=\bm{I_n}\\
  \therefore (\bm{I_{n}-AB})^{-1}&=\bm{I_{n}}+\bm{A}(\bm{I_{p}-BA})^{-1}\bm{B}
  .
  \end{align*}
  
  
     
\section{Problem 3}
Question:\\
Verify that the Moore-Penrose inverse $\bm{A^{+}}$ of a symmetric
matrix $\bm{A}$ is symmetric. (Thus, $\bm{A^{+}}$ is also a
Moore-Penrose inverse of $\bm{A}$)\\
Proof:\\
\begin{align*}
  &\bm{AA^{+}A}=A\Rightarrow \bm{AA^{+^{\prime}}A}=\bm{A}\\
  &\bm{A^{+}AA^{+}}=\bm{A^{+}}\Rightarrow \bm{A^{+^{\prime}}AA^{+^{\prime}}}=\bm{A^{+^{\prime}}}\\
  &(\bm{AA^{+}})^{\prime}=\bm{AA^{+}}\Rightarrow \bm{A^{+^{\prime}}A}=(\bm{A^{+^{\prime}}A})^{\prime}\\
  &(\bm{A^{+}A})^{\prime}=\bm{A^{+}A}\Rightarrow \bm{AA^{+^{\prime}}}=(\bm{AA^{+^{\prime}}})^{\prime}
  .
  \end{align*}
  
\section{Problem 4}
Question:\\
Use the fact that $r(\bm{AB})\leq \min(r(\bm{A}), r(\bm{B}))$, show that\\
\begin{align*}
  r(\bm{A})=r(\bm{PA}), r(\bm{A})=r(\bm{AQ})
  .
  \end{align*}\\
if $\bm{P}$ and $\bm{Q}$ are invertible matrices.\\
Proof:\\
\begin{align*}
  &r(\bm{PA})\leq \min(r(\bm{A}),r(\bm{P}))\leq r(\bm{A})\\
  &\because \bm{A}=(\bm{P})^{-1}\bm{PA}\\
  &\therefore r(\bm{A})=r((\bm{P})^{-1}\bm{PA})\leq
  \min(r((\bm{P})^{-1}),r(\bm{PA}))\leq r(\bm{PA})\\
  &\therefore r(\bm{A})=r(\bm{PA})\text{ if }\bm{P}\text{ is
    invertible matrix}
  .
  \end{align*}
  
\begin{align*}
  &r(\bm{AQ})\leq \min(r(\bm{A}),r(\bm{Q}))\leq r(\bm{A})\\
  &\because \bm{A}=\bm{AQ} (\bm{Q})^{-1}\\
  &\therefore r(\bm{A})=r(\bm{AQ}(\bm{Q})^{-1})\leq
  \min(r((\bm{Q})^{-1}),r(\bm{AQ}))\leq r(\bm{AQ})\\
  &\therefore r(\bm{A})=r(\bm{AQ})\text{ if }\bm{Q}\text{ is
    invertible matrix}
  .
  \end{align*}
  
      
          
\end{document}
